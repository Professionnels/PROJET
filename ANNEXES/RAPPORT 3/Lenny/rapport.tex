\documentclass[12pt]{article}
\usepackage[francais]{babel}
\usepackage[UTF8]{inputenc}
\usepackage[T1]{fontenc}
\usepackage{times}
\usepackage{graphicx}
\usepackage{fancyhdr}
\usepackage{eurosym}
\usepackage{color}
\usepackage{soul}
\usepackage[ left = 4.5cm, right = 3.5cm]{geometry}


\pagestyle{fancyplain} \chead{}\lhead{\textit{Les Professionnels}} \rhead{\emph{\textit{Evasion}}}

\definecolor{pseudorouge}{RGB}{200, 50, 50}
\definecolor{pseudoblue}{RGB}{20,10,230}
\definecolor{texteGris}{RGB}{50,50,75}

\begin{document}
\thispagestyle{empty}
\begin{center}
\fontsize{21}{21}{\textbf{Rapport de Soutenance 2 \vspace*{0.2cm}}}

\end{center}

\vspace*{0.7cm}

\begin{center}
\fontsize{21}{21}{\textbf{- Les Professionnels/}}
\fontsize{21}{21}{\textbf{2013-2014 -}}
\end{center}

\vspace*{0.5cm}

\begin{center}
\includegraphics[scale=01.0]{evasion}
\end{center}

\vspace*{0.5cm}

\fontsize{14}{14}
\begin{center}
{Lenny \textcolor{pseudorouge}{\textit{"Le Noob"}} Danino - danino\_l}
\end{center}
\begin{center}
Louis \textcolor{pseudoblue}{\textit{"El Parain"}} Kédémos - kedemo\_l
\end{center}
\begin{center}
Anatole \textcolor{pseudoblue}{\textit{"Totonut"}} Moreau - moreau\_a
\end{center}
\begin{center}
Khalis Chalabi - chalab\_k
\end{center}

\begin{center}
\includegraphics[scale=00.20]{infini}
\end{center}


\setlength{\headheight}{13pt} % Haut de page
\setlength{\headsep}{2.5cm} % Entre le haut de page et le texte
\setlength{\footskip}{2.5cm}

\newpage
\thispagestyle{empty}
\pagestyle{fancyplain} \chead{}\lhead{\textit{Les Professionnels}} \rhead{\emph{\textit{Evasion}}}
\tableofcontents

\newpage
\setcounter{page}{1} 
\section{Introduction}

C’est la fin d’une année de travail et nous Les Professionnels écrivons ce dernier rapport avec une belle vision sur notre projet mais surtout avec une grande fierté. Il est vrai que plus d’une fois nous fumes fatigués et poussés à bout pour différentes raisons car ce projet est un véritable marathon. Il est pourvu de nombreuses étapes mais nous pouvons dire que nous l’avons d’ores et déjà gagné car nous sommes arrivés au bout.\\

Chaque soutenance avait sa dose de difficultés et de regrets mais la seconde plus particulièrement car nous n’avions pu ajouter certaines idées que nous souhaitions montrer. Heureusement cette troisième soutenance vient compléter la dernière et aujourd’hui notre jeu est complet. Nous avons donc rajouté principalement les collisions et un vrai multijoueur en réseau. Ce sont nos deux points forts de cette soutenance mais d’autres seront présentés dans ce rapport.\\

Il n’y a pas eu seulement des difficultés mais aussi des très bons moments. En effet, passer toutes les vacances et tous les jours  d’une année avec les mêmes personnes, forcément cela crée des liens qui dureront. Chacun d’entre nous a travaillé selon ses moyens et ses capacités et même si des déséquilibres de niveaux en informatique ont parfois posé problèmes, tout le monde a apporté sa participation. C’est le principal but d’un projet en commun : savoir s’entendre avec les autres comme nous le ferions en entreprise.\\

Aujourd’hui nous sommes fiers du résultat  de notre jeu et encore plus de pouvoir le montrer. Il reflète ce que nous attendions depuis le début de l’année mais il montre aussi à quel point nous nous sommes améliorés. Principalement en pratique et en code mais aussi dans notre attitude. Nous sommes devenus plus sérieux, plus attentifs et plus à l'écoute des avis des autres. Les avancés, retard et problèmes rencontrés sont tous écrit dans ce rapport pour constater et comprendre notre progression, en parallèle avec le cahier des charges.\\

\newpage
\setcounter{page}{1} 
\section{Lenny Danino}

\subsection{Présentation}

blabla

\newpage
\thispagestyle{empty}
\pagestyle{fancyplain} \chead{}\lhead{\textit{Les Professionnels}} \rhead{\emph{\textit{Evasion}}}
\listoffigures

\newpage
\setcounter{page}{1} 
\section{Conclusion}

C’est donc la fin de notre projet pour cette année, nous espérons qu’il plaira aux joueurs. Nous avons tous passés énormément de temps dessus et le résultat est qu’il dépasse nos espérances. Il respecte notre vision principale du début de l’année mais il contient beaucoup plus d’idées qu’à l’origine. La charge de travail ne fit qu’augmenter mais ce fut toujours avec défi que nous continuâmes  notre jeu vidéo.  Nous prenons plaisir aujourd’hui à jouer aux quelques niveaux que notre jeu possède. Il est certain qu’en dehors des cours nous continuerons à améliorer notre jeu.\\

Comme dans chaque travail d’équipe il nous arriva de rencontrer différents malentendus et quelques désaccords et retards mais aussi de très bonnes avancés et de découvertes. En effet pour notre site web nous avons appris le HTML ; pour les graphiques nous avons appris Blender et Photoshop ; pour le code en lui-même nous avons appris XNA et l’implémentation de celui-ci. Ainsi nous ressortons enrichis par ce projet à tous les niveaux : socialement et dans les compétences aussi.\\

\end{document}
