\documentclass[12pt]{article}
\usepackage[francais]{babel}
\usepackage[UTF8]{inputenc}
\usepackage[T1]{fontenc}
\usepackage{times}
\usepackage{graphicx}
\usepackage{fancyhdr}
\usepackage{eurosym}
\usepackage{color}
\usepackage{soul}
\usepackage[ left = 4.5cm, right = 3.5cm]{geometry}


\pagestyle{fancyplain} \chead{}\lhead{\textit{Les Professionnels}} \rhead{\emph{\textit{Evasion}}}

\definecolor{pseudorouge}{RGB}{200, 50, 50}
\definecolor{pseudoblue}{RGB}{20,10,230}
\definecolor{texteGris}{RGB}{50,50,75}

\begin{document}
\thispagestyle{empty}
\begin{center}
\fontsize{21}{21}{\textbf{Rapport de Soutenance 2 \vspace*{0.2cm}}}

\end{center}

\vspace*{0.7cm}

\begin{center}
\fontsize{21}{21}{\textbf{- Les Professionnels/}}
\fontsize{21}{21}{\textbf{2013-2014 -}}
\end{center}

\vspace*{0.5cm}

\begin{center}
\includegraphics[scale=01.0]{evasion}
\end{center}

\vspace*{0.5cm}

\fontsize{14}{14}
\begin{center}
{Lenny \textcolor{pseudorouge}{\textit{"Le Noob"}} Danino - danino\_l}
\end{center}
\begin{center}
Louis \textcolor{pseudoblue}{\textit{"El Parain"}} Kédémos - kedemo\_l
\end{center}
\begin{center}
Anatole \textcolor{pseudoblue}{\textit{"Totonut"}} Moreau - moreau\_a
\end{center}
\begin{center}
Khalis Chalabi - chalab\_k
\end{center}

\begin{center}
\includegraphics[scale=00.20]{infini}
\end{center}


\setlength{\headheight}{13pt} % Haut de page
\setlength{\headsep}{2.5cm} % Entre le haut de page et le texte
\setlength{\footskip}{2.5cm}

\newpage
\thispagestyle{empty}
\pagestyle{fancyplain} \chead{}\lhead{\textit{Les Professionnels}} \rhead{\emph{\textit{Evasion}}}
\tableofcontents

\newpage
\setcounter{page}{1} 
\section{Introduction}

C’est la fin d’une année de travail et nous Les Professionnels écrivons ce dernier rapport avec une belle vision sur notre projet mais surtout avec une grande fierté. Il est vrai que plus d’une fois nous fumes fatigués et poussés à bout pour différentes raisons car ce projet est un véritable marathon. Il est pourvu de nombreuses étapes mais nous pouvons dire que nous l’avons d’ores et déjà gagné car nous sommes arrivés au bout.\\

Chaque soutenance avait sa dose de difficultés et de regrets mais la seconde plus particulièrement car nous n’avions pu ajouter certaines idées que nous souhaitions montrer. Heureusement cette troisième soutenance vient compléter la dernière et aujourd’hui notre jeu est complet. Nous avons donc rajouté principalement les collisions et un vrai multijoueur en réseau. Ce sont nos deux points forts de cette soutenance mais d’autres seront présentés dans ce rapport.\\

Il n’y a pas eu seulement des difficultés mais aussi des très bons moments. En effet, passer toutes les vacances et tous les jours  d’une année avec les mêmes personnes, forcément cela crée des liens qui dureront. Chacun d’entre nous a travaillé selon ses moyens et ses capacités et même si des déséquilibres de niveaux en informatique ont parfois posé problèmes, tout le monde a apporté sa participation. C’est le principal but d’un projet en commun : savoir s’entendre avec les autres comme nous le ferions en entreprise.\\

Aujourd’hui nous sommes fiers du résultat  de notre jeu et encore plus de pouvoir le montrer. Il reflète ce que nous attendions depuis le début de l’année mais il montre aussi à quel point nous nous sommes améliorés. Principalement en pratique et en code mais aussi dans notre attitude. Nous sommes devenus plus sérieux, plus attentifs et plus à l'écoute des avis des autres. Les avancés, retard et problèmes rencontrés sont tous écrit dans ce rapport pour constater et comprendre notre progression, en parallèle avec le cahier des charges.\\

\newpage
\setcounter{page}{1} 
\section{Lenny Danino}

\subsection{Présentation}

J’ai intégré l’Epita car mon principal intérêt lorsque j’étais au lycée était l’informatique. Avant cela je n’avais jamais touché aux langages tels que le Caml, le C++ ou le C Sharp mais je trouvais réellement intéressant la capacité d’écrire quelque chose que l’ordinateur comprend et de le faire apparaitre à l’écran. Bien sûr je me suis rapidement rendu compte qu’être informaticien c’est bien plus que cela. Aussi cela s’est vu à mon niveau et j’ai eu vraiment peur que cela me pénalise dans le choix d’un groupe.\\

Durant le séminaire en début d’année je me suis fait des amis qui malgré notre différence de classe m’ont accepté dans leur groupe sans même regarder mes capacités. Ils cherchaient plus à créer un groupe solide et qui peut s’améliorer qu’un groupe disparate. En plus de mon retard en informatique, un groupe de projet à quatre est assez difficile à contrôler. Il faut faire en sorte que tout le monde soit d’accord pour avancer dans le projet mais puisque nous nous entendions bien, ce ne fut pas la partie la plus difficile durant cette année.\\

En effet à partir du moment où nous devions travailler notre projet en même  temps que nos examens du premier semestre ainsi que du second, le partage pour l’organisation des taches est devenu plus compliqué. Personnellement je me devais absolument de réussir les soutenances car elles me permettent de compenser avec mes notes en IP.//

Ce projet m’a donc permis de m’investir dans mes lacunes et de me faire progresser tout en apprenant la cohésion de groupe et ce que c’est de participer à un projet commun. J’en ressors plus sur de moi et plus au courant de ce qui m’attend dans la vie professionnel.\\

\newpage
\subsection{Réalisations}
\subsubsection{Première soutenance}

A cause des partiels de fin d’année qui furent très important pour moi je ne me suis pas beaucoup investi pour cette troisième soutenance. Je vais donc résumer ce que j’ai fait durant la première et la seconde soutenance et terminer par les quelques améliorations pour la dernière.\\

Puisque j’avais des difficultés de compréhension du code, j’ai préféré m’investir sur tout ce qui n’en nécessitait que très peu. Tout au long de l'année je me suis donc occupé du Latex et des rapports de soutnances. J ai plusieurs fois demande des conseils sur la mise en page à des amis en ing1 qui ont eu de bonnes notes. Je pense que cela a apporté une qualité plus professionnelle à notre jeu. Aussii, je me suis occupé des classes décors et personnage et de leurs attributs.\\ 

 Pour cela j'ai du apprendre à configurer une classe de base et certains TP m'ont bien aidé. Notre 3D ne fonctionnant pas encore à ce stade du jeu je ne pouvais pas visualiser mon avancée mais cela me paraissait bon et mes camarades me l’ont fait comprendre.\\

De plus j’ai participé à la première version de notre site web avec Anatole. Le premier résultat n'avais pas plu mais nous fimes mieux lors de la seconde soutenance. J’ai donc dû apprendre quelques bases en HTML aussi pour avancer dans mon cahier des charges ainsi que du CSS.\\

Cela peut paraitre peu pour une première soutenance mais pour mon niveau c’était vraiment capital et compliqué. J’ai pu me concentrer pour la seconde soutenance.\\

\newpage
\subsubsection{Seconde soutenance}

La première chose que j’ai faite lorsque je me suis remis dans mon travail pour le projet fut de reprendre les classes que j’avais faites et de voir si je pouvais les améliorer ou si je pouvais les modifier pour qu’elles ressemblent plus au code que les autres avaient fait. Par ailleurs j’avais décidé de participer aussi au son du jeu. J’ai donc cherché plusieurs bruitages et sons d’ambiance pour parfaire l’atmosphère assez discrète que nous cherchions à reproduire. Je me suis aussi interessé au multijoueur car cela donne une nouvelle dimension à notre jeu et je voulais vraiment pouvoir y jouer avec d'autres amis.\\

J’ai cherché sur pas mal de tutoriels sur internet pour me renseigner sur le double écran par exemple. J’ai montré ce que j’avais appris aux autres et même si cela n’aida pas beaucoup je me réjouis d’avoir participé pour cette fonction. Cela m’a aussi amené à regarder le réseau. Principalement je devais m’en occuper entièrement mais le mieux que je pouvais faire était un réseau pour un jeu en console or le nôtre ne fonctionne pas du tout de cette manière-là. Même en lisant les livres expliquant le CSharp et XNA je ne réussis pas cette tâche. J’ai donc dû demander de l’aide aux autres pour la troisième soutenance et ensemble nous avons pu  parvenir à réaliser un réseau pour notre jeu.\\

Aussi j'ai créé plusieurs textures pour les décors et pour les personnages. J'ai appris à utiliser Blender et Photoshop ce qui ne fut pas aisé tout de suite mais j'étais doué pour la représentation spatiale et cela fut tres utile pour la 3D de nos personnages qui sont cubiques.\\

Ainsi on a pu déjà à ce stade du jeu recréer une petite partie et nous fument tous très satisfait du résultat qui dépasait nos attentes. Surement que cela était dû au fait qu'au lieu d'inventer nos textures nous reprenions celles que nous trouvions sur internet et les arrangeait selon nos envies.\\

\newpage
\subsubsection{Troisième soutenance} 

Apres avoir passé tous nos examens, j'ai commencé à regarder la procédure d'installation et de désinstallation de notre jeu. Plusieurs logiciels permettent de faire ça aujourd'hui et même windows directement. J'ai donc fait en sorte de pouvoir installer facilement notre jeu pour que beaucoup de joueurs puissent y jouer.\\

J’ai participé avec Anatole à l’éditeur de  maps qui est réellement impressionnant. Il est possible en quelques cliques d’ajouter les personnages que nous souhaitons aux endroits que nous voulons, et de faire un labyrinthe  comme parcours.\\

Je compte  faire apparaitre le site et tous les éléments qui doivent normalement accompagner notre jeu. Cependant les joueurs pourront sélectionner ou désélectionner par exemple les langues qu’ils ne souhaitent pas ou les graphismes qu’ils ne veulent pas. Je sais par exemple que pour des joeurs anglophones il n'y a pas d'interêts pour eux qu'ils installent la langue française.\\

Aussi j'ai aidé à la création du réseau de notre jeu avec Louis puisque seul j'avais vraiment du mal. Nous ne sommes pas sur de terminer à temps mais plus nous passons du temps dessus plus on se rapproche de ce qu'attendent les joueurs. La création d'unclient et d'un serveur est plutôt bien géré avec XNA et puisque cela fait un an que nous es dessus, c'est un vrai atout.\\

\newpage
\subsubsection{Problèmes rencontrés}

Comme je l’ai souvent écrit, ma principale difficulté fut de ne pas savoir programmer et d’avoir déjà des lacunes comparativement aux autres. Cependant il y eu aussi le problème des classes. En effet je ne partage pas les mêmes horaires que mes camarades donc il y a eu des fois où je fus obligé de rater une réunion pour parler du projet comme d’autres fois où en attendant qu’ils sortent de leur heures de cours je m’avançais sur mon travail.\\

Pour les personnages j’ai du attendre en début d’année que Louis termine la 3D pour pouvoir visualiser mon travail. Aussi lorsque je créais les textures du prisonnier et du gendarme, il a d’abord fallu que j’apprenne  les outils Blender et Photoshop. Ce fut interessant de se représenter mon personnage avant de le voir en mouvement et complet puisque je faisaiembre par membre.  Je sais qu’aujourd’hui j’ai la capacité pour refaire d’autres textures. Je pense qu’après la soutenance j’en rajouterais par plaisir pour notre jeu.\\

Pour le réseau je sais que c’est surement la partie qui donne le plus de difficultés pour chaque groupe mais j’avais réellement envie de le faire. Je me sentais un peu en inégalité concernant mon travail accompli et celui des autres. J’ai passé plusieurs jours à ne faire que lire des tutoriels sur internet pour apprendre les serveurs et les clients. J’ai voulu ajouter un moyen de communiquer entre les joueurs connectés par exemple un chat pour rendre le jeu plus interactif et intéressant.\\

\newpage
\subsection{Conclusion}

Je suis donc très satisfait de mon avancée cette année et j'espère que notre jeu plaira. J ai beaucoup apprécié mon groupe malgré quelques divergences  parfois mais je pense sincèrement qu'elles sont la preuve de notre attachement pour ce projet.\\

Mon niveau personnel en programmation a bien augmenté et j'en suis satisfait. Je remercie mes camarades pour cette année et cette idée partagée ainsi que les professeurs qui m'ont soutenu des que je perdais confiance en moi. Plusieurs fois c'est pendant leurs cours que des idées me sont venues et ils furent de bons conseils, fort de leurs experiences.\\

 J'espère qu'un jour nous retravaillerons ensemble  mes camarades et moi sur des  projets tout aussi interessants.\\

\newpage
\thispagestyle{empty}
\pagestyle{fancyplain} \chead{}\lhead{\textit{Les Professionnels}} \rhead{\emph{\textit{Evasion}}}
\listoffigures

\newpage
\setcounter{page}{1} 
\section{Conclusion}

C’est donc la fin de notre projet pour cette année, nous espérons qu’il plaira aux joueurs. Nous avons tous passés énormément de temps dessus et le résultat est qu’il dépasse nos espérances. Il respecte notre vision principale du début de l’année mais il contient beaucoup plus d’idées qu’à l’origine. La charge de travail ne fit qu’augmenter mais ce fut toujours avec défi que nous continuâmes  notre jeu vidéo.  Nous prenons plaisir aujourd’hui à jouer aux quelques niveaux que notre jeu possède. Il est certain qu’en dehors des cours nous continuerons à améliorer notre jeu.\\

Comme dans chaque travail d’équipe il nous arriva de rencontrer différents malentendus et quelques désaccords et retards mais aussi de très bonnes avancés et de découvertes. En effet pour notre site web nous avons appris le HTML ; pour les graphiques nous avons appris Blender et Photoshop ; pour le code en lui-même nous avons appris XNA et l’implémentation de celui-ci. Ainsi nous ressortons enrichis par ce projet à tous les niveaux : socialement et dans les compétences aussi.\\

\end{document}
