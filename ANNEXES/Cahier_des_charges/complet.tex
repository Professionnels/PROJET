\documentclass{report}
\usepackage[francais]{babel}
\usepackage[UTF8]{inputenc}
\usepackage[T1]{fontenc}
\usepackage{graphicx}
\usepackage{fancyhdr}
\usepackage{eurosym}
\usepackage{color}
\usepackage{soul}
\usepackage{lscape}

\pagestyle{fancyplain} \chead{}\lhead{\textit{Les Professionels}} \rhead{\emph{\textit{Evasion}}}

\definecolor{pseudorouge}{RGB}{200, 50, 50}
\definecolor{pseudoblue}{RGB}{20,10,230}

\begin{document}
\thispagestyle{empty}
\begin{center}
<<<<<<< HEAD
 \fontsize{40}{40}{\textbf{Cahier des charges\vspace*{0.2cm}\newline\textit{Evasion}}}
=======
 \fontsize{40}{40}{\textbf{Cahier des charges en plus \vspace*{0.2cm}\newline\textit{Evasion}}}
>>>>>>> completlouis
\end{center}

\vspace*{0.7cm}

\begin{center}
 \fontsize{21}{21}{\textbf{- Les Professionels/}}
 \fontsize{21}{21}{\textbf{2013-2014 -}}
\end{center}

\vspace*{0.5cm}

\begin{center}
\includegraphics[scale=01.0]{evasion}
\end{center}

\vspace*{0.5cm}

\fontsize{14}{14}
\begin{center}
{Lenny \textcolor{pseudorouge}{\textit{"Le Noob"}} Danino - danino\_l}
\end{center}
\begin{center}
Louis \textcolor{pseudoblue}{\textit{"El Parain"}} Kédémos - kedemo\_l
\end{center}
\begin{center}
 Anatole \textcolor{pseudoblue}{\textit{"Totonut"}} Moreau - moreau\_a
\end{center}
\begin{center}
Khalis Chalabi - chalab\_k
\end{center}

\begin{center}
\includegraphics[scale=00.20]{infini}
\end{center}



\newpage

\pagenumbering{arabic}

\thispagestyle{empty}
\tableofcontents

\newpage



\section[Introduction]{Introduction}

\subsection[Création du groupe]{Création du groupe}
Nous \emph{Les Professionnels}, jeunes codeurs que nous sommes, avons le plaisir de vous présenter notre projet informatique de première année à \bsc{epita} que nous aurons travaillé sur une durée de six mois.\\

Provenant de différents lycées, nous ne nous connaissions pas avant notre arrivée à \bsc{epita} cette année. Le séminaire a permis à deux d’entre nous de faire connaissance et les deux autres se sont connus en classe. Faisant parti de la même classe, une bonne entente entre nous vit rapidement le jour et la décision de faire un groupe ensemble s’est vite imposée à nous lorsque nous avons partagé nos points de vue très similaires pour le projet. Créer un jeu vidéo, à qui cela ne plait-il pas !?\\


	Cela parait facile mais la liste de travail à faire nous a rapidement démoralisée (mais l’âme du Gamer fût plus forte !). Heureusement, le choix du type de jeu que nous voulions c'est-à-dire un « Beat’em Up » nous a rebooté. Etant fans de séries telles que Prison Break notre objectif était de suivre un scenario similaire à l’histoire et donc de s’évader d’une prison (d’où Evasion !). Aussi La fierté d’être les créateurs d’un jeu vidéo est un bon encouragement pour fournir un travail soigné et intéressant.\\

	C’est donc ainsi que nous  avons formé notre groupe des Professionnels avec en objectif la réussite de celui-ci bien sûr, mais surtout avec une entraide digne de véritables camarades russes !\\

\subsection[Les Membres]{Les Membres}
\vspace{0.5cm}

\subsubsection*{Lenny Danino - \textcolor{pseudorouge}{\textit{"Le Noob"}}}

Bonjour je me présente, Lenny DANINO 20 ans, étudiant en première année à EPITA et depuis toujours grand fan de jeux vidéo. Autant vous le dire tout de suite, je suis donc exalté a l’idée de créer un jeu !! Des tas d’idées me traversent l’esprit et j’espère qu’elles plairont à mes camarades pour pouvoir toutes les réaliser. J'apprécie vraiment le fait de travailler en groupe car je pense que cela est une très bonne expérience pour l’avenir.\\

	Je n’ai commencé à coder que cette année mais je ne désespère pas car l’informatique reste une passion malgré mes lacunes. Maintenant que j’ai la possibilité de participer au développement d’un jeu je compte m’y plonger entièrement et pourquoi pas un jour montrer à ma famille le jeu comme le travail fourni derrière. Cela me permettrait de leur donner un aperçu de ce qu'est coder !\\

	Je pense que j’ai ma place dans ce groupe car il faut (un vrai leader !!) une personne pour s’occuper des lignes de codes qui paraissent insignifiantes mais qui restent importantes tout de même !! De plus je compte bien ajouter ma part de plaisanterie et de création à ce projet. En soit, nous \emph{Les Professionnels} sommes une équipe de passionnés avant tout !

\subsubsection*{Khalis Chalabi}

Alors voilà vous avez déjà dû me remarquer, je suis le seul qui se distingue des trois autres membres qui ont voulu faire comme tout le monde en se choisissant un pseudonyme. Je me présente quand même Khalis alias Khalis, pas de surnom pour moi parce qu'il n'y en a aucun qui me convient et parce que je trouve que Khalis c'est la classe comme prénom. Unique en son genre et inoubliable !\\
\newline
Contrairement à la majorité de l'école je ne suis vraiment pas un geek, désolé de vous décevoir. J'aime voir autre chose que l'écran de mon ordinateur rempli de lignes de codes incompréhensibles, de poneys aux couleurs de l'arc-en-ciel ou encore affchant constamment un jeu vidéo. Sinon à part ça, l'informatique m'intéresse vraiment (et oui, informatique ne rime pas qu'avec geek)c'est un domaine qui évolue constamment et qui est de nos jours incontournable. Depuis le lycée j'ai décidé que je travaillerai dans ce domaine plus tard sans m'aventurer cependant dans la jungle qu'est l'informatique. Je m'y suis donc mis cette année et il est grand temps maintenant de créer LE jeu qui va révolutionner le monde : \textit{Evasion}. Petit problème, la programmation et moi ce n'est pas trop ça pour l'instant...\\
\newline
Mais je ne me fait pas trop de soucis, vous avez vu le nom de mon groupe je pense que ca en dit long sur nous et puis je suis quelqu'un qui apprend plutôt vite. Donc rendez-vous à la soutenance finale avec un jeu révolutionnaire fait par en partie des non-geeks !

\subsubsection*{Anatole Moreau -  \textcolor{pseudoblue}{\textit{Totonut}}}

Passionné d'informatique depuis trois ans, l'intérêt n'est pas ce qu'il me manque (j'ai un intérêt de la même nature pour l'univers de la musique, donc les prises de tête, je connais). La 3D est un nouvel univers pour moi, et j'ai très (très) hâte d'y toucher, alors je vais donner tout ce que j'ai pour ce projet. 
Durant mon année de terminale, pendant que mes "collègues" lycéens travaillaient peut-être un peu plus, j'était déjà plongé dans des projets personnels, et aussi dans la solitude (des gens qui codent des jeux en C, ça court pas les rues). Alors cette première année à l'\bsc{epita} est un tournant dans ma passion pour l'informatique, car cette fois c'est un projet avec d'autres passionnés, et ça s'annonce passionnant. J'espère vous rendre compte de mes intentions et de celle du groupe de projet dans lequel je me suis accoutumé, dans un résultat final qui devrait être à la hauteur !
Un beau voyage en perspective !


\subsubsection*{Louis Kédémos -  \textcolor{pseudoblue}{\textit{El Parrain}}}

Bonjour ! Travailler sur ce projet est pour moi quelque chose d'important. C'est la première fois que je réalise un projet d'une telle ampleur. Je fais de l'informatique depuis de nombreuses années. J'ai eu la chance de toucher à presque tous les aspects (même de l'assembleur, si si !). Je me suis mis en groupe avec Lenny, Anatole et Khalis car je sais qu'ensemble, nous arriverons à faire un bon jeu. Particulièrement avec Khalis. Nous savons tous qu'il est un véritable meneur d'Hommes.\\

Malheureusement, ses capacités en programmation (catastrophiques) ne lui permettent pas d'être chef de projet. C'est moi qui ai pris ce poste, non pas grâce à mes connaissances, mais grâce à l'emprise que j'ai au sein du groupe. À vrai dire, il n'y a pas photo. Je suis le meilleur du groupe, dans tous les domaines. Non, en fait, je suis le meilleur d'\bsc{epita}, mais ne le dites à personne. J'aurais pu faire le jeu tout seul, mais étant altruiste, je me fais aider par des petits nouveaux dans le domaine pour les faire progresser.\\

J'aimerais bien dire blague à part, mais ce n'est pas une blague et ma pré- sentation touche à sa fin.\\

Fin.


\newpage


\section[Le Projet]{Le Projet}

\subsection[Présentation]{Présentation}
Notre groupe s'est constitué assez tôt,  durant le mois d'octobre. Nous savions que l'on voulait faire un jeu vidéo mais nous ne savions pas quel type de jeu. Nous nous sommes vu plusieurs fois au cours des mois suivants et un membre de notre groupe à eu une idée qui a plu directement à tous les autres membres. C'est à ce moment là qu'est né \textit{Evasion}.\\\\

Le jeu que nous voulons créer va en fait s'inspirer d'une série américaine : \textit{Prison Break}. Dans cette série, le personnage principal veut sauver son frère de prison et pour cela il se fait incarcérer. Une fois évadés, ils leur arrivent beaucoup d'aventures. Notre jeu serait donc un jeu à la première personne. Le but de ce jeu serait de s'évader de la prison avec le plus de discrétion possible. Une fois sortie de prison il faudrait réussir à quitter le pays en se procurant de l'argent, de nouveaux passeports... pour échapper aux forces de polices.\\\\

C'est lors de nos réunions que nous sommes parvenus à décider quel jeu nous voulions faire. En discutant et en partageant nos avis nous en sommes arrivés à un nom de jeu : \textit{Evasion} et à un scénario. Une fois cette étape franchie, nous avons commencé à réfléchir à ce qu'il y aurait dans le jeu : armes, outils, personnages, décors... Nous avions fait une liste de tous ces éléments, ce qui nous a facilité la tâche lorsque nous nous sommes penchés sur les classes dont nous aurions besoin. Nous avons donc listé les classes dont nous aurions besoin pour la conception de notre jeu vidéo ainsi que leurs propiétés et leurs méthodes. Dès lors que nous avions trouvé le type de jeu que nous voulions faire et un début de scénario, les idées coulèrent à flots et beaucoup de questions s'installèrent dans nos esprit. Ce jeu vidéo va nous demander énormément de travail et d'apprentissage car nous n'avons pas toutes les connaissances requises.

\subsection[L'intérêt du Projet]{L'intérêt du Projet}

Ce projet est avant tout caractérisé par un but pédagogique, l'essentiel étant de trouver une idée qui nous plaît, pour ne pas perdre notre motivation. Nous visons donc à développer un jeu jouable et pas trop poussé au niveau réalisme, pour assurer un beau résultat dans ce premier projet de groupe. Nous avons ainsi choisi un jeu dont le scénario nous a séduit, ainsi que le gameplay. Ce projet va pouvoir s'adresser aux fans de jeux à la première personne, ainsi qu'aux personnes attirées par les missions d'infiltration.\\

 Tout d'abord, nous avons traversé ensemble une phase de création très enrichissante. Nous avons laissé parler notre imagination en écrivant nos idées sur le thème du jeu sur une feuille vierge. Au fur et à mesure, la feuille s'est remplie et a très vite été remplie. Nous avons passé une deuxième étape : le tri. Les idées réalisables et essentielles ont été gardées, et nous avons essayé de limiter les éléments facultatifs. Le jeu que l'on a conçu va être une mise à l'épreuve de notre aptitude à structurer un code. Notre jeu va posséder un nombre considérable d'objets, et de communications entre eux via des pointeurs. Nous avons aussi décidé d'utiliser des décors en trois dimensions, ce qui va être l'essentiel de notre travail pour ce projet. En effet, certains membres de notre groupe possèdent déjà une petite expérience dans la programmation de jeux vidéos, et l'apparition de la 3D dans ce projet est un objectif intéressant pour tout le monde. Chacun d'entre nous va donc se voir confronté à plusieurs problèmes, et nous allons tous y apporter des solutions. De plus, nous avons affaire pour la première fois à un travail collectif, ce qui implique une réelle préparation et une réelle réflexion avant d'entamer le projet. Nous avons réparti les tâches de chacun en fonction des points forts et faibles des membres du groupe. Nous allons tous apprendre de l'expérience des autres dans ce projet. C'est enfin un avant goût des tâches en entreprise dans un temps limité, avec certaines restrictions (ici données par le cahier des charges).

\newpage

\section[Découpage du Projet]{Découpage du Projet}

Pour réaliser notre jeu, nous allons répartir les tâches. Notre but serait d'obtenir un jeu en 3D. Cela suppose d'utiliser un moteur 3D et physique pour la création de l'univers du jeu. Nous souhaiterions aussi implémenter des effets sonores. \\

Lenny et Khalis développeront la partie de la gestion interne du jeu. Ils seront charges de coder toute les classes nécessaires a la création de l'univers de jeu. Ils créeront toutes les classes personnages, toutes les classes objets et toutes les classes du décor. 

Chaque personnage (héros, ennemi, personnage non joueur) est capable d'utiliser un objet. Il faudra créer les interactions entre chaque objet et les personnages, entre les objets et les murs et les personnages et les murs. De plus, les personnages auront chacun leurs propre vie, résistance, vitesse, caractéristiques\ldots\\

Anatole et Louis se chargeront de la partie graphique. C'est-à-dire qu'ils s'occuperont de l'affichage du jeu, de la composition de la fenêtre du jeu (personnages, objets, décors\ldots), du menu et de tout ce qui est nécessaire a un rendu visuel. Ils devront  notamment créer tous les personnages, objets et décors à l'aide du logiciel de création 3D libre \bsc{blender}.\\

Tous les quatre, nous développeront ensemble la gestion à proprement parler du jeu : la boucle principale. Celle-ci devra adapter le déroulement du jeu en fonction des actions du ou des joueurs. Cette partie nous prendra la majeure partie du temps de développement. Le joueur sera en mesure de contrôler son personnage avec les touches du clavier, de tourner la caméra avec la souris et d'utiliser les objets qu'il aura récolter. Le joueur pourra aussi accéder au menu pause. De là, il pourra sauvegarder, quitter ou accéder aux options du jeu.

De plus, chaque action du joueur aura des répercussions sur le reste du jeu. En se déplaçant, le personnage produit du bruit qui attire les ennemis. Il faudra donc créer les ennemis de telle sorte qu'ils puissent réagir à l'environnement sonore. Sans stimuli particulier, les ennemis devront quand même se déplacer aléatoirement dans le niveau. En clair, il faudra coder une \bsc{I.A.} fonctionnelle.\\

Toute la partie code se fera à l'aide du \bsc{C\#} et de Microsoft Visual Studio . Nous utiliserons en plus le Framework \bsc{xna} qui gère les graphismes 2D et 3D. De plus, \bsc{xna} supporte aussi l'implémentation de shaders. Ceux-ci permettent le rendu d'effets graphiques. Nous les utiliserons pour créer une atmosphère particulière dans notre jeu. 

Comme mentionné précédemment, \bsc{blender} sera aussi utilisé. Pour créer l'identité visuelle du jeu et les textures nécessaires, nous utiliserons \bsc{Photoshop}.\\

Pendant le développement, nous essayerons de rester le plus modulaire possible. Chaque classe, chaque fonction devra être le plus indépendant possible des autres, tout en conservant une certaine homogénéité. Ainsi, chaque membre du groupe pourra développer la partie qui lui incombe sans se préoccuper de l'avancement de ses partenaires. 

La modification de certains bout de code ou l'ajout de nouvelles fonctionnalités sera aussi nettement plus simple. La partie la plus difficile étant la boucle principale. Ajouter de nouveaux objets, de nouveaux personnages ou de nouveaux niveaux n'en sera que plus facile.\\

Un site web sera aussi développé. Il permettra de rendre compte de l'avancement du projet. Sera présent sur ce site nos pistes de réflexions, des croquis, des images tests. En bref, tous les éléments rattachés au jeu. Il contiendra aussi nos rapports de soutenance. Bien sûr, le jeu pourra être téléchargé soit à l'état final, soit dans ses versions intermédiaires, pour s'assurer de sa portabilité vers différents ordinateurs.



\newpage
\thispagestyle{empty}

\addcontentsline{toc}{section}{4 Planing du Projet}
\begin{landscape}

\begin{tabular}{|c|p{6cm}|p{6cm}|p{6cm}|}


\hline
 & & & \\
   & \emph{Première soutenance} &\emph{ Deuxième soutenance} & \emph{Troisième soutenance}\\ 
& & & \\
\hline




Anatole Moreau & 
\begin{itemize}
\item[-]  Codage de l’affichage et de l’animation 3D
\end{itemize}
 &   
\begin{itemize}
\item[-] Création du site web
\item[-] Recherche d’un environnement sonore (bruitages, musique de fond)
\item[-] Mise en place de la boucle principale
\end{itemize}
&
\begin{itemize}
\item[-] Finalisation du site web
\item[-] Finalisation de l’environnement sonore
\end{itemize}
\\
\hline




Lenny Danino & 
\begin{itemize}
\item[-] Codage des classes de base du décor : mur et porte
\end{itemize}
 &   
\begin{itemize}
\item[-] Interaction entre les différents éléments du jeu : utilisation d’un objet ou d’une arme
\item[-] Codage des autres éléments du décor : passages secrets, coffre, portes améliorées
\item[-] Mise en place de la boucle principale
\end{itemize}
&
\begin{itemize}
\item[-]  Finalisation des interactions entre les éléments du jeu
\end{itemize}
\\
\hline




Khalis Chalabi & 
\begin{itemize}
\item[-] Codage des classes personnages : héros, ennemis, personnages non joueurs
\item[-] Rapport de la première soutenance en \LaTeX
\end{itemize}
 &   
\begin{itemize}
\item[-] Codage des objets : armes, outils
\item[-] Création du site web
\item[-] Mise en place de la boucle principale
\item[-] Rapport de la deuxième soutenance en \LaTeX
\end{itemize}
&
\begin{itemize}
\item[-] Finalisation du site web
\item[-] Rapport de la troisième soutenance en \LaTeX
\end{itemize}
\\
\hline



Louis Kédémos & 
\begin{itemize}
\item[-]  Création des personnages 3D
\item[-]  Création du menu d’accueil
\end{itemize}
 &   
\begin{itemize}
\item[-] Codage du système de collision en 3D
\item[-] Création des décors et objets en 3D
\item[-] Mise en place de la boucle principale
\end{itemize}
&
\begin{itemize}
\item[-]  Finalisation du système de collision
\end{itemize}
\\
\hline



\end{tabular}
\end{landscape}


\newpage

\section[Conclusion]{Conclusion}

A première vue, notre projet pourrait paraître faramineux. Mais nous sommes motivés. Motivés par l'envie de voir nos idées concrétister. Motivés par l'envie de mettre à l'épreuve notre apprentissage. Mais avant tout, motivés par une envie de se dépasser. \\

Réaliser un tel projet en groupe est impossible s'il n'existe aucune communication entre les différents membres. Heureusement, nous avons tous \bsc{Facebook} et des portables avec un forfait sms illimité. \\

Beaucoup de travail, beaucoup de nuits blanches, beaucoup de stress en perspective\@. Mais, on ne s'appelle pas \emph{Les Professionnels} pour rien.
\end{document}
