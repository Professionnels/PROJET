\documentclass{article}
\usepackage[francais]{babel}
\usepackage[UTF8]{inputenc}
\usepackage[T1]{fontenc}
\usepackage{graphicx}
\usepackage{fancyhdr}
\usepackage{eurosym}
\usepackage{color}
\usepackage{soul}

\pagestyle{fancyplain} \chead{}\lhead{\textit{Les Professionnels}} \rhead{\emph{\textit{Evasion}}}

\definecolor{pseudorouge}{RGB}{200, 50, 50}
\definecolor{pseudoblue}{RGB}{20,10,230}

\begin{document}
\thispagestyle{empty}
\begin{center}
 \fontsize{21}{21}{\textbf{Rapport de Soutenance 1 \vspace*{0.2cm}\newline\textit{Evasion}}}
\end{center}

\vspace*{0.7cm}

\begin{center}
 \fontsize{21}{21}{\textbf{- Les Professionels/}}
 \fontsize{21}{21}{\textbf{2013-2014 -}}
\end{center}

\vspace*{0.5cm}

\begin{center}
\includegraphics[scale=01.0]{evasion}
\end{center}

\vspace*{0.5cm}

\fontsize{14}{14}
\begin{center}
{Lenny \textcolor{pseudorouge}{\textit{"Le Noob"}} Danino - danino\_l}
\end{center}
\begin{center}
Louis \textcolor{pseudoblue}{\textit{"El Parain"}} Kédémos - kedemo\_l
\end{center}
\begin{center}
 Anatole \textcolor{pseudoblue}{\textit{"Totonut"}} Moreau - moreau\_a
\end{center}
\begin{center}
Khalis Chalabi - chalab\_k
\end{center}

\begin{center}
\includegraphics[scale=00.20]{infini}
\end{center}

\newpage
\thispagestyle{empty}
\tableofcontents

\newpage
\fontsize{12}{12}
\pagenumbering{arabic}
\section{Introduction}

\par
Cela fait un bon moment que nous attendions de pouvoir nous concentrer a fond sur notre projet et finalement nous y voila. Nous avions commence deja avant la fin de l'annee mais maintenant que les partiels et les controles sont passes l'occasion de prendre de l'avance pour le projet nous est possible. Cela permettra de revoir les erreurs ou modifications plus tot et d'ameliorer notre tres estime jeu.
\newline

\par
Comme nous l'avions dis dans notre cahier des charges, notre jeu consistera en un Beat'em Up dans lequel notre personnage principal, un prisonnier qui n'aurait jamais du l'etre,  s'echappe de sa prison et tente de rentrer chez lui par n'importe quel moyen et donc rencontre des gardes et des ennemis  qui le bloqueront dans son evasion. Ce jeu sera en vue 3D a la troisieme personne.
\newline

\par
Ainsi puisque notre gourmandise nous amena a une creation d'un autre niveau, il a fallu apprendre a utiliser des outils precis comme BLENDER mais surtout a bien manier le C\# et XNA ce qui ne fut pas tres facile pour tous mais les tutoriels ont ete tres pratiquent. Aujourd'hui encore des difficultes subsistent mais de bonnes ameliorations furent realisees.
\newline

\par
Par ailleurs, il a fallu nous organiser comme nous ne sommes pas dans les memes classes. C'est donc pour cela que nous utilisons GITHUB qui permet de voir l'ensemble de la progression du jeu mais aussi de recuperer le code et cela independemment de l'endroit ou l'on se trouve. Aussi nous avons monte un site internet sur notre jeu ou nous mettons l evolution de celui-ci, les progressions, les noms de ceux qui participent aux differentes parties et bien plus viendra par la suite.Nous restons constamment en contact, et nous partageons nos fichiers. Notre code est également organisé en sections et sous sections ce qui permet de coder de manière plus efficace tout en etant  plus lisible.
\newline

\par
Notre jeu n'est pas encore jouable mais le menu est visible et cliquable. Il est donc loin d'etre termine mais on peut deja avoir une vision de ce a quoi il ressemblera. Nos efforts finiront par etre recompenses mais pas encore!
\newline

\par
Parlons du groupe maintenant. Nous sommes une veritable equipe ou chacun peut compter sur les autres pour l'aider quelque soit le probleme. Malgre le fait que nos classes ne soient pas les memes nous faisons tout pour se voir un maximum et de partager nos idees sur le jeu ou meme sur tout et n'importe quoi !
\newline

\par
Voici donc le travail que nos avons effectue depuis nos debuts sur le projet et celui que nous executerons pour la prochaine soutenance.

\newpage

\subsection{danino\_l \textcolor{blue}{"Le Noob"}}
\subsubsection{Personnages/Decors}

\par
\underline{Les attributs des personnages}
\newline

\par

\underline{Les differents decors}
\newline


\end{document}